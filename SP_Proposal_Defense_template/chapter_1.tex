%   Filename    : chapter_1.tex 
\chapter{Introduction}
\label{sec:researchdesc}    %labels help you reference sections of your document

\section{Overview}
\label{sec:overview}

The Philippines is a global center of marine biodiversity and has established aquaculture as a significant contributor to total fishery production \cite{aypa2000, bfar2019}. The country produces over 4 million tonnes of seafood annually and considered to be the 11th largest producer in the world. Aquaculture is deeply integrated into Filipinos' livelihoods, encompassing fish cultivation and the production of various aquatic commodities, including mollusks. Among these are blood clams or cockles \textit{(Tegillarca granosa)} which hold considerable economic and environmental significance.

Maintaining a balanced male-to-female ratio of blood cockles is crucial to prevent over-harvesting and ensure sustainable production because an imbalanced ratio can lead to over-exploitation and can impact the population's sustainability. However, there is limited literature on \textit{T. granosa} that has a thorough understanding of its sex-determining mechanisms, particularly concerning sexual dimorphism in morphological and morphometric characteristics \cite{breton2017sex}.

Currently, sex determination methods for blood cockles are invasive, including dissection, and histological examinations which often result in the death of specimens. While there is growing literature on aquaculture commodities sex identification using machine and deep learning, there is a notable scarcity of research specifically addressing \textit{T. granosa} \cite{miranda2023}.

This study, titled "A Non-Invasive Sex Identification of \textit{T. granosa} using Machine Learning," aims to provide a comprehensive analysis of blood cockles by leveraging their morphological and morphometric characteristics. By integrating machine learning and computer vision techniques, the study seeks to identify distinct features that indicate sexual dimorphism between male and female blood cockles.

\section{Problem Statement}

Accurately identifying the sex of \textit{T. granosa} is important in order to promote sustainable aquaculture and biodiversity by maintaining a balanced male-to-female ratio. A balanced ratio helps prevent over-harvesting. Although sex identification is important for blood cockle population management and sustainable aquaculture, there is a notable lack of research in creating non-invasive methods to identify the sex of \textit{T. granosa}. Many of the latest studies and approaches are based on invasive methods like dissection or histological analysis, which are impractical for large-scale aquaculture operations focused on conservation.

The existing invasive methods for identifying the sex of \textit{T. granosa} often require dissection, a technique that involves cutting open the shell to visually inspect the gonads \cite{erica2018}. This causes harm and death to the specimens. In some cases, histological examination is used to examine tissue samples through a microscope, leading to further destruction of the organism \cite{may2021}. These methods are time-consuming, labor-intensive, and can pose a threat to population management, especially when it is essential to maintain a balanced sex ratio for breeding programs. Moreover, invasive methods also require technical skills to execute properly. Aquaculture operations, particularly in resource-limited settings, face challenges in accessing laboratory equipment like microscopes and staining tools which complicates the process.

A less invasive approach employed by aquaculturists is to monitor spawning behavior in which individuals are separated and stimulated to reproduce in order to determine their sex through the release of gametes \cite{miranda2023}. Although it is indeed less invasive than dissection, spawning still involves inducing stress in blood cockles and may not be completely effective for fast identification in large populations.

Given the limitations of both invasive and non-invasive methods highlight the need for a more advanced approach. An alternative, non-invasive method involving machine and deep learning technologies might solve these issues by providing a fast, accurate, and effective solution without harming or stressing the blood cockles.

\section{Research Objectives}
\label{sec:researchobjectives}

\subsection{General Objective}
\label{sec:generalobjective}

The general objective of this study is to develop a non-invasive method for identifying the sex of \textit{Tegillarca granosa} using machine and deep learning integrated with computer vision technologies. This method aims to provide accurate and streamlined sex identification without causing harm to the specimens, thus supporting sustainable aquaculture practices.

\subsection{Specific Objectives}
\label{sec:specificobjectives}

To achieve the general objective of developing a non-invasive sex identification of \textit{T. granosa} using machine and deep learning, the following specific objectives have been established:

\begin{enumerate}
   \item To collect and organize a comprehensive datasets of \textit{T. granosa} which will include high-quality images and relevant morphological measurements that will serve as the basis for machine-learning model.

   \item To develop and implement machine learning models that can classify the sex of \textit{T. granosa} based on the collected dataset, implementing algorithms such as support vector machines (SVM) for pre-evaluation, and deep learning models such as Squeezenet and Unet. 
   
   \item To evaluate the performance of the models used using performances metrics such as accuracy, precision, recall, and F1-score to ensure the effectiveness and reliability of the models.
   
   \item To compare the developed models against existing methods, such as dissection and spawning, and assess their potential for real-world application in aquaculture operations.
\end{enumerate}


\section{Scope and Limitations of the Research}
\label{sec:scopelimitations}

This study focuses on developing a non-invasive method for identifying the sex of \textit{Tegillarca granosa} using machine learning, deep learning, and computer vision technologies. The goal is to provide an accurate and efficient means of sex identification without causing harm to the specimens, contributing to sustainable aquaculture practices.

The researchers will work with 500 spawned blood cockles taken from Panay island, specifically Zarraga Iloilo and Ivisan Capiz, equally divided between 250 males and females, obtained by spawning induced through temperature shock. The researchers will personally gather linear measurements, including length, width, height, rib count, length of the hinge line, and distance between the umbos using the vernier caliper. Images and corresponding views of the specimens will also be collected by the researchers under the supervision of the University Researchers Associate from the Institute of Aquaculture, College of Fisheries and Ocean Sciences.

Data collection will take place at the hatchery facility of the University of the Philippines Visayas. Data gathering will be conducted in batches, depending on the availability of spawned samples.

The method developed in this study is specific to \textit{Tegillarca granosa} and may not be generalized to other species. The model is trained exclusively for \textit{Tegillarca granosa} and  morphological features including length, width, height, rib count, length of the hinge line, and distance between the umbos may not be shared by other shellfish species. 

\section{Significance of the Research}
\label{sec:significance}

This study will give significant advancement in non-invasive sex identification methods in \textit{T.granosa} providing innovative solutions that could solve the challenges in identifying sex  and reshape approaches to aquaculture. Thus, the significance of this study extends to the following:

 \textit{Research Institution.} The result of this study focusing on the sex-identification mechanism of bivalves, specifically \textit{Tegillarca granosa}, will provide valuable insights into universities and research centers that focus on fisheries and coastal management such as the UPV Institute of Agriculture that aim to develop sustainable development and suitable culture techniques.

 \textit{Fishermen.} By developing a non-invasive method in sex identification, this study can help long-term harvest efficiency and maintain the ratio of the harvest which can help prevent over-exploitation of the \textit{T. granosa.} and maintaining sustainable ratio between male and female. 

 \textit{Coastal Communities.} The result of this study would be beneficial for the coastal communities that are reliant on their source of income with aquaculture commodities like blood cockles. Maintaining the diversity and aspect ratio of male and female may increase the market value of blood cockle production since it faces significant obstacles worldwide due to the fluctuating seed supplies and scarcity of broodstock from the wild. 

 \textit{Future Researchers.} The result of this study would serve as the basis for studies that involve sex identification in bivalves such as \textit{T. granosa}. Some technologies are yet to be explored in machine learning, deep learning, and computer vision technologies that can lead to higher accuracy and distinguish the presence of sexual dimorphism in the \textit{T. granosa}.


